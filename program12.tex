\documentclass[6pt,a4paper]{report} 
\usepackage{amsmath} 
\usepackage{amsfonts} 
\usepackage{amssymb} 
\usepackage{graphicx} 
\usepackage[left=3cm,right=3cm,top=2cm,bottom=2cm]{geometry} 
\author{Sujan VL} 
\title{About my Creation} 
\begin{document} 
\maketitle 
\chapter{themes Creating} 
\section*{What is Creating?} 
"\textbf{Creating}" Creating is making new things that have not existed before, usually by combining two or three things that existed. Art is almost always called creative, but so are other activities like  "\textit{music, mathematics, technology, business, craft, building, gardening and forestry}",if they combine things to get new things.

When several people cooperate in more than one process of creating, that is a "\textit{creative network}". This cuts risk since anything creative involves some chance that it will simply not work.

What is "not creative" is usually called imitative, and includes anything that is mostly trying not to do what has never been done – like law, sports, science, war, and especially education. These involve more direct competition, so risk comes from the competitor who may do it better. The word team is used for such groups.

Most people fear change and new things, and so what is most stable and considered important by society tends to be imitative and done in teams. But unless society encourages creating it loses out to others by competition in business or technology, so there must be some creative networks, somewhere, in labs or art studios.

Views of control of creative work also depend very much on religious and spiritual views of nature (or a creator) creating Man, according to economist Lester Thurow. Those ethical traditions that see "\textit{Man in the image of God}" have created legal codes (see guild, intellectual right, intellectual interest, intellectual property) to tightly control what they are creating or have created. These are now the basis of organizations such as WIPO or ICANN. Often, such laws also create control over things which are merely found, not created.

Many people give away what they have created and never think about it as property – other people try to control it and get paid every time anyone sees it – or even every time they just talk about it.
\section*{The Creation Definition} 
Creating is making new things that have not existed before, usually by combining two or three things that existed. Art is almost always called creative, but so are other activities like music, mathematics, technology, business, craft, building, gardening and forestry, if they combine things to get new things. \\
\textbf{The creating verb(used without object} \\
\begin{itemize} 
\item to cause to come into being, as something unique that would not naturally evolve or that is not made by ordinary processes. 
\item to evolve from one's own thought or imagination, as a work of art or an invention.
Synonyms: invent, contrive, devise, initiate, originate
\item Theater. to perform (a role) for the first time or in the first production of a play. 
\item to cause to happen; bring about; arrange, as by intention or design:
to create a revolution; to create an opportunity to ask for a raise.
\end{itemize} 
Similar to conceive and spawn and the exact opposite of destroy, create is a word that often implies a little bit of imagination. In fact, it takes a lot of creativity to create something spectacular; that is, unless you're a robot, and then your creations occur automatically. Or Mother Nature, where creation just happens naturally: Birds create nests, the tides create waves, and snowstorms create days off from school. \\

\chapter{Free fire-SUJAN} 
\begin{small} 
\section*{Guns lists} 
\subsection*{Gloowall lists} 
\begin{itemize} 
\item black wall 
\item anime wall
\item green wall 
\item Red wall 
\end{itemize} 
\subsection*{free fire  Teams} 
\begin{itemize} 
\item karnataka Premier League-KPL
\begin{itemize} 
\item kalabhairava 
\item Sujan 
\end{itemize} 
\item UG ayush 
\begin{itemize} 
\item Gyan sujan 
\item Gyan rishi 
\end{itemize} 
\item Gyan rithi 
\begin{itemize} 
\item Yama yodha 
\item Yama captian
\end{itemize} 
\end{itemize} 
\section*{Match lists} 
\subsection*{KLP Season 1 Rankings} 
\begin{enumerate} 
\item SJL 
\item KA mafia
\item Taguru Gaming
\end{enumerate} 
\subsection*{teams are participate in KPL} 
\begin{enumerate} 
\item yamadhutaru 
\begin{enumerate} 
\item gyan gaming 
\item ungraduate gamer 
\item sooneeta 
\end{enumerate} 
\item Mg kannadiga 
\begin{enumerate} 
\item Hassan 
\item Arkalgud 
\item Konanur
\end{enumerate} 
\end{enumerate} 
\end{small}
\end{document}